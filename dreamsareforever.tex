\documentclass[a4paper, 11pt]{article} % Font size (can be 10pt, 11pt or 12pt) and paper size (remove a4paper for US letter paper)

\usepackage[protrusion=true,expansion=true]{microtype}

\usepackage[T1]{fontenc}
\linespread{1.05}

\makeatletter

\renewcommand{\maketitle}{ % Customize the title - do not edit title and author name here, see the TITLE block below
\begin{flushright} % Right align
	{\LARGE\@title} % Increase the font size of the title

	\vspace{40pt} % Some vertical space between the title and author name

	{\large\@author} % Author name
	\\\@date % Date

	\vspace{30pt} % Some vertical space between the author block and abstract
\end{flushright}
}

%----------------------------------------------------------------------------------------
%	TITLE
%----------------------------------------------------------------------------------------

\title{\linespread{2.1}\textbf{These Dreams are Forever}\\ % Title
Janelle Mon\'ae's Metropolis as Historiography of Black Feminism} % Subtitle

\author{\textsc{Grady Ward} % Author
\\{\textit{Brandeis University}}} % Institution

\date{\today} % Date

%----------------------------------------------------------------------------------------

\begin{document}

\maketitle % Print the title section

%----------------------------------------------------------------------------------------
%	ABSTRACT
%----------------------------------------------------------------------------------------

\begin{abstract}

Janelle Mon\'ae's \emph{Metropolis Suite} offers its listeners a view into a dystopian world that marries elements of futurism with past and current conversations about oppression and freedom.
The analogy of black folk as androids is central, pervasive and interchangeable. 
Like the concept of enslavement, androids were `constructed' in the minds of their masters to serve capitalistic systems, and their masters are (over time) forced to come to terms with the fact that something much more human lies underneath their abstraction as machinery. 
Lyrically, stylistically, in the impassioned ballads and in the small details, Janelle Mon\'ae practices and educates her listenership about Black Feminism, showing how the liberation of the most oppressed creates routes for the liberation of all.
Moreover, Mon\'ae calls us all to action toward a better future with a carefully hedged politic of radicalism and self-empowerment.

\end{abstract}

\vspace{30pt}

%----------------------------------------------------------------------------------------
%	ESSAY BODY
%----------------------------------------------------------------------------------------

\section*{Introduction}

Janelle Mon\'ae caught my attention with her voice, she captured my mind with her lyrics, and she won my heart with her politics.
Mon\'ae is an American musician, mogul, actress and songwriter from Kansas City, Kansas.
The majority of her discography consists of \emph{Metropolis}, a three-album musical drama divided into five suites, which blends the lines between music, fable and politics.
With shoe-tapping dance hall anthems, operatic ballads, unapologetic rap, dissonant punk and shimmering orchestral overtures, the \emph{Metropolis Suite} cannot be surmised succinctly in style nor tone.
It follows the story of an android (Cindi Mayweather) who falls in love with a human (Anthony Greendown) and is sentenced to death for this crime\cite{wolfmasters}.
Cindi escapes her initial surprise to become a fugitive\cite{manymoons}, then an underground musical icon\cite{favoritefugitive}, then a society shaking radical\cite{fiveseveneighttwoone}, fighting for her freedom and the freedom of her people through myth and music.

Evoking images and ideas from Fritz Lang's Weimar classic by the same name, \emph{Metropolis} centers and shares the same themes of class, love and revolution\cite{metropolis}.
Additionally, though both have central imagery of mechanical women, Mon\'ae questions the the construction of women by their society: ``I'm a product of metal? I'm a product of the man."\cite{happyhunting}. 
Rather than having mechanical women as a means of controlling the proletariat, as in Lang's work\cite{metropolis}, Mon\'ae's Android heroine realizes her mission and capability to win the freedom of the oppressed.
Though both \emph{Metropolis} works center around protagonists learning about how capitalism is predicated on the exploitation of the opressed, their plots and lenses could not be more different.
Unlike in Lang's work, Mon\'ae centers the Woman as the revolutionary leader, rather than than filling that role by dispossessed aristocracy.
Unlike in Lang's work, Mon\'ae realizes freedom for her characters on their own terms, by their own means, without the need for establishment validation.
Unlike in Lang's work, Mon\'ae's heroine saves herself.

Though these contrasts are a starting place for a discussion of Mon\'ae's work, they mainly begin to shed light on how Mon\'ae's choices are painstakingly deliberate.
It can be difficult to consider the profound consequences that Mon\'ae's \emph{Metropolis} presents, often so buried underneath layers of distracting musical excellence.
But make no mistake: in \emph{Metropolis} Mon\'ae has chosen to rewrite a classic.
She is making a deliberate and strategic descision to center an othered woman as the focal point of social and radical change.
She is taking an established masterpiece of world cinema, and challenging not only its plot, but its epistemology.

The \emph{Metropolis Suite} is embodied Black Feminism.
Through fiction and musical fusion, Mon\'ae has the freedom and power to construct elaborate parables that are opaque and transparent, varried but consistent.
Moreover, by tracing and naming Black Feminist ideas and motifs through the three albums, we see an intentional and deliberate historiography of Black Feminism, through its three waves.
In this paper, we will discuss how Mon\'ae crafts a story which mirrors the history of Black Feminism, we will show how she uses the standpoint of an `ordinary' android to demonstrate the extraordinary potential of the oppressed individual, and we will enumerate how her music empowers and educates her listeners on Black Feminism without ever relying on theory.
Mon\'ae paints the Android a path to humanity, the dispossessed a path toward establishment, and the oppressed a path to freedom.
We will trace out these paths. 

%------------------------------------------------

\section*{The Chase Suite: Voice, Violence and Dissemblance}
- Escaping physical violence
``You know the rules!"\cite{wolfmasters}
``You know the rules!"\cite{happyhunting}
NAMES Cindi as 57821, not by her self assigned name. 
Connection to Lynching.

- Place of Black Women in many contexts (Many Moons)
``A pretty face makes a pretty picture, but fall in love and they are coming to get you"\cite{happyhunting}
``And when the world just treats you wrong // just come with me and I'll take you home" \cite{manymoons}
``I keep my feet on solid ground. Freedom! Your free but in your mind, Your freedom's in a bind!"\cite{manymoons}

- Mr. President as a self-critical look at the society, cyclicness of poverty, respectablitiy poltics (as a means toward opression, mothers)
Cycle of Poverty
``Fuel is running low, and I've got places to go"\cite{mrpresident}
``Ghettos keep a crying out to streets full of zombies // Kids are killing kids and then the kids join the army"\cite{danceordie}
``These kids round' killin each other, they lost they minds, they gone. They quittin' school, making babies and can barely read. Some gone off to their fall, lord have mercy on them."\cite{sincerelyjane}

- Deference to the rights and voices of black men as an established norm
``Let me just hear ya squeak."\cite{happyhunting}


- Smile (Dissemblance!)
``Smile, though your heart is aching // Smile, even though it's breaking."\cite{smile}
``Smile, through all fear and sorrow // Smile and may be tomrrow // You'll find the sun come shining through // For you."\cite{smile}
``You must keep on trying // Smile, whats the use of crying // You'll find that life is still worthwhile // If you just smile."\cite{smile}
``I know I'm mysterious sometimes, and you are mysterious too."\cite{primetime}

%------------------------------------------------

\section*{The ArchAndroid Suite II: Performance and Women growing to own Activism}

``When you get elevated, they love it or they hate it... They tryin' to take all of your dreams but you can't allow it."\cite{danceordie}
``Just keep rebelling away, you gotta dream it away // Because the weatherman on TV ain't creating the weather."\cite{danceordie}

``Run on for your life or you can dance you can die // She's praying in the sand like she's the last samurai"\cite{danceordie}

``Call me weak, or better yet you can call me // you can call me your hero, baby."\cite{faster}

Women as leaders of activism
``This man wants to see another jubilation //
And that man wants us in emancipation //
And then there is the man who wants a stronger nation //
You see we really got to and i think that we ought to //
Protect the mind from degradation //
Sow in the seeds of education //
They run from us, are we that dangerous? //
There's a war in all the streets and yes the freaks must dance or die!"\cite{danceordie}

Locked Inside --> Abusive Relationship.
``I'm locked inside a land called foolish pride // where the man is always right // he hates to talk but loves to fight // is that alright?"\cite{lockedinside}

``I'm trying to find my peace.  I was made to believe theres something wrong with me!"\cite{coldwar}
``This is a cold war, do you know what you're fighting for."\cite{coldwar}



%------------------------------------------------

\section*{The ArchAndroid Suite III: Naming Love and its Limits}

``Tell me are you bold enough to reach for love?"\cite{manymoons}

- Love, original sin, opening her eyes to the world.

``This time I shall be unafraid // And violence will not move me"\cite{babopbyeya}
``I see beyond tomorrow // This life of strife and sorrow // My freedom calls and I must go."\cite{babopbyeya}
``Love is such a novelty // A rarely painted masterpiece // A place few people go or ever know"\cite{sayyouwillgo}
``Sir Greendown // Let's leave in an hour // Meet me at the tower // Ride your horse"\cite{greendown}

%------------------------------------------------

\section*{The Electric Lady Suite IV: Queerness, Performance and Activism}

``They keep us underground working hard for the greedy, but when it's time pay they turn around and call us needy."\cite{queen}

- Rejection of Violence, unite in love
``I use my words when stones come around.''\cite{manymoons}
In an interlude, `Bop Bot Say What' proposes violence as a means to redress the agreed upon inconsistiences of ruling logic: ``hitting us all up in the head and wondering why we don't think straight"\cite{chromeshoppe}.
``I'm praying for the man sitting without much time // May he understand the clock will never rewind // Wisen him and sharpen him and give him a motto // hate no more, said he must hate no more"\cite{danceordie}
``Sudanese and Congolese who put the roll in the rock // From here to Sudan, Metropolis to Iraq"\cite{danceordie}

- Activism
``March to the streets 'cause I'm willing and I'm able"\cite{queen}
``Melanie 45221, Assatta, 8550."\cite{chromeshoppe}
``Keep leading like a young Harriet Tubman"\cite{queen}
``Day dreamers please wake up, we can't sleep no more."\cite{sincerelyjane}
``Electric Ladies will you sleep? Or will you preach?"\cite{queen}

- Queerness
``Would your got accept me in my black and white? Would he approve the way I'm made? Or should I reprogram reprogram and get down?!?"\cite{queen}


%------------------------------------------------

\section*{The Electric Lady Suite V: Dissemblance, Self Empowerment and Victory}

Empowerment: 
``Oh make it rain, ain't a thang and the sky to fall (The silver bullet's in your hand and the war's heating up)" \cite{manymoons}
``She'd keep it to herself and nobody could understand her // Even when she thought that she couldn't she carried on // She couldn't imagine both of her daughters here all alone"\cite{ghettowoman}
``I wish they could just realize // That all you've ever needed was someone to free your mind // Carry on Ghetto Woman // Cause even in your darkest hours I still see your light" \cite{ghettowoman}

Non-reliance on Dieties:
``A long long way to find the one // We'll keep on dancing till she comes // These dreams are forever"\cite{danceordie}
``Heaven is betting on us."\cite{primetime}

%------------------------------------------------

\section*{Meta Analysis: Self Naming, Performance, Epistomology}

- Self Naming and categorization
``She who writes the movie owns the script and the sequel."\cite{queen}
Labels, Categories
Mon\'ae has frequently worked with an Atlanta based group, Jaspects.
Their collaborations, including \emph{My First Love}, \emph{Peachtree Blues} and \emph{2012}, feature Mon\'ae's stunning voice over the band's signature modern jazz sound\cite{peachtreeblues}\cite{mytruelove}.
In her 2009 collaboration with the Atlanta group Jaspects, Mon\'ae's song 2012 implored listeners to ask what freedom means to them, and how they actualize it for themselves and for others.
Though she is not featured on the title song of the album ``The Polkadotted Stripe", freedom within the context of the album revolves around the idea of false definitions.
``We see black or white, polkadots or stripes. Freedom we deny, lets open our eyes."\cite{polkadottedstripe}
Mon\'ae questions dichotomies and categorization throughout the Metropolis suite, through explict and subtle means.
Though not always as clear as in Q.U.E.E.N. (``Categorize me, I defy Every Label"\cite{queen}), 
On her own style, Mon\'ae refuses to adhere to categorization which undermines the potency of her work:
``We?re growing up in the iPod generation in terms of genres. I do away with labels and categories. I don't believe in them ... That was created by man."\cite{joeyguerra2010}


``Categorize me, I defy every label."\cite{queen}
``I Just want to be, as free as I can be."\cite{polkadottedstripe}
``We see black or white, polkadots or stripes. Freedom we deny, lets open our eyes."\cite{polkadottedstripe}

- Interactions with Whiteness
ALL OF \cite{favoritefugitive}.

- Episotomology
``Add us to equations but they'll never make us equal."\cite{queen}
``Because the booty don't lie"\cite{queen}


- Calls to Wake up
``Day dreamers please wake up, we can't sleep no more."\cite{sincerelyjane}
``Electric Ladies will you sleep? Or will you preach?"\cite{queen}

%------------------------------------------------


\section*{Misc/Unassigned}


%----------------------------------------------------------------------------------------
%	BIBLIOGRAPHY
%----------------------------------------------------------------------------------------

\nocite{*}
\bibliography{ArchAndroid,Metropolis,ElectricLady,ClassReading,External}{}
\bibliographystyle{unsrt}

%----------------------------------------------------------------------------------------

\end{document}