\documentclass[a4paper, 11pt]{article} % Font size (can be 10pt, 11pt or 12pt) and paper size (remove a4paper for US letter paper)

\usepackage[protrusion=true,expansion=true]{microtype}

\usepackage[T1]{fontenc}
\linespread{1.05}

\makeatletter

\renewcommand{\maketitle}{ % Customize the title - do not edit title and author name here, see the TITLE block below
\begin{flushright} % Right align
	{\LARGE\@title} % Increase the font size of the title

	\vspace{40pt} % Some vertical space between the title and author name

	{\large\@author} % Author name
	\\\@date % Date

	\vspace{30pt} % Some vertical space between the author block and abstract
\end{flushright}
}

%----------------------------------------------------------------------------------------
%	TITLE
%----------------------------------------------------------------------------------------

\title{\textbf{\linespread{2.1}These Dreams are Forever}\\ % Title
Janelle Mon\'ae's Metropolis as a Historiography of Black Feminism} % Subtitle

\author{\textsc{Grady Ward} % Author
\\{\textit{Brandeis University}}} % Institution

\date{\today} % Date

%----------------------------------------------------------------------------------------

\begin{document}

\maketitle % Print the title section

%----------------------------------------------------------------------------------------
%	ABSTRACT
%----------------------------------------------------------------------------------------

\begin{abstract}

Janelle Mon\'ae's \emph{Metropolis Suite} marries elements of futurism with past and current conversations about oppression and freedom through the lens of Black Feminism.
Lyrically, stylistically, in the impassioned ballads and in the small details, Janelle Mon\'ae practices and educates her listenership about Black Feminism, showing how the liberation of the most oppressed creates routes for the liberation of all and how love is a radical practice.
Moreover, Mon\'ae calls us all to action toward a better future with a carefully hedged politic of radicalism and self-empowerment.
The \emph{Metropolis Suite} should be read like a syllabus: a clear and coordinated set of conversations with scholarship, in the exact terms of quotidian and extraordinary experiences that shape the Black Woman's standpoint.
\end{abstract}

\vspace{30pt}

%----------------------------------------------------------------------------------------
%	ESSAY BODY
%----------------------------------------------------------------------------------------

\section*{Introduction}

Janelle Mon\'ae caught my attention with her voice, she captured my mind with her lyrics, and she won my heart with her politics.
Mon\'ae is an American musician, mogul, actress and songwriter from Kansas City, Kansas.
The majority of her discography(thus far) consists of \emph{Metropolis}, a three-album musical drama divided into five suites, which blends the lines between music, fable and politics.
With shoe-tapping dance hall anthems, operatic ballads, unapologetic rap, dissonant punk and shimmering orchestral overtures, the \emph{Metropolis Suite} cannot be surmised succinctly in style nor tone.
It follows the story of an android (Cindi Mayweather) who falls in love with a human (Anthony Greendown) and is sentenced to death for this crime \cite{wolfmasters}.
Cindi escapes her initial surprise to become a fugitive \cite{manymoons}, then an underground musical icon \cite{favoritefugitive}, then a society shaking radical \cite{fiveseveneighttwoone}, fighting for her freedom and the freedom of her people through myth and music.

The analogy of black folk as androids is central, pervasive and interchangeable. 
Like the system of enslavement, androids were `constructed' in the minds of their masters to serve capitalistic systems, and their masters are (over time) forced to come to terms with the fact that something much more human lies underneath their abstraction as machinery.
Evoking images and ideas from Fritz Lang's Weimar classic by the same name, \emph{Metropolis} centers and shares the same themes of class, love and revolution \cite{metropolis}.
Additionally, though both have central imagery of mechanical women, Mon\'ae questions the the construction of women by their society: ``I'm a product of metal? I'm a product of the man." \cite{happyhunting}. 
Rather than having mechanical women as a means of controlling the proletariat, as in Lang's work \cite{metropolis}, Mon\'ae's Android heroine realizes her mission and capability to win the freedom of the oppressed.
Though both \emph{Metropolis} works center around protagonists learning about how capitalism is predicated on the exploitation of the oppressed, their plots and lenses could not be more different.

Unlike in Lang's work, Mon\'ae centers the Woman as the revolutionary leader, rather than than filling that role by dispossessed aristocracy.
Unlike in Lang's work, Mon\'ae realizes freedom for her characters on their own terms, by their own means, without the need for establishment validation.
Unlike in Lang's work, Mon\'ae's heroine is not physically controlled by her creator, nor is she a substitution for a `real' human.
Unlike in Lang's work, Mon\'ae's heroine saves herself.
Though these contrasts are a starting place for a discussion of Mon\'ae's work, they mainly begin to shed light on how Mon\'ae's choices are painstakingly deliberate.
% It can be difficult to consider the profound consequences that Mon\'ae's \emph{Metropolis} presents, often so buried underneath layers of distracting musical excellence.
% But make no mistake: in \emph{Metropolis} Mon\'ae has chosen to rewrite a classic.
% She is making a deliberate and strategic decision to center an othered woman as the focal point of social and radical change.
She is taking an established masterpiece of world cinema, and challenging not only its plot, but its epistemology.

The \emph{Metropolis Suite} is embodied Black Feminism.
Through fiction and musical fusion, Mon\'ae has the freedom and power to construct illustrate parables that are opaque and transparent, varried but consistent.
Moreover, by tracing and naming Black Feminist ideas and motifs through the three albums, we see an intentional and deliberate historiography of Black Feminism, through its three waves (and three albums).
In this paper, we will discuss how Mon\'ae crafts a story which mirrors the history of Black Feminism, we will show how she uses the standpoint of an `ordinary' android to demonstrate the extraordinary potential of the oppressed individual, and we will enumerate how her music empowers and educates her listeners on Black Feminism without ever relying on theory.
Mon\'ae paints the Android a path to humanity, the dispossessed a path toward establishment and recognition, and the oppressed a path to freedom.
We will trace out these paths. 

%------------------------------------------------

\section*{The Chase Suite: Voice, Violence and Dissemblance}

``You know the rules!" \cite{wolfmasters}, the singsong establishment chides.
Reasoning is devoid from the call to violence, only adherence to established protocol dictating the nature of love and humanity.
\emph{The Chase Suite}, the first album and (and the first suite) of \emph{Metropolis} sets the stage for Cindi Mayweather's awakening.
Rather than by her chosen name, she is introduced by her serial identification number, 57821, an early connection to a black feminist practice of self naming. 
Audre Lorde describes this tradition in her essay \emph{Scratching the Surface}: ``If we do not define ourselves for ourselves, we will be defined by others-for their use and to our detriment'' \cite{lordescratching}.
Mayweather's choice to call herself Cindi reclaims her humanity (as faced labor) from a system which has named her for its use as impersonal capitol.

Mayweather has fallen `desperately in love' with a human, Anthony Greendown, a crime in the eyes of the state, and the gleeful narrator is announcing her punishment: a chase for her capture and destruction.
Between the arcane and specific rules issued for her capture (`chainsaws and electro-daggers!' \cite{wolfmasters}), and the casual tone, the implications of slavery are clear, particularly within the context of the enslaved and the slaveholding. 
Less explicit are the ties to the pleasure that violence and suppression bring out in the exertion of power over others, and the compliancy and responsibility of the (white) female narrator in perpetuating this violence.
We should not forget that postcards of lynchings are not a feature of the archaic past, and violence predicated on sexual identity and preference is still very much alive in our world today.

In the world of \emph{Metropolis}, love is consequential.
In bell hooks' treatise on love, power and relationships \emph{All About Love: New Visions}, a message which resonated with me was the idea that love cannot exist within a power imbalance; that the fundamental nature of love as a verb, as a practice of care and intimacy, places as a prerequisite the full and equal humanity and dignity of its participants \cite{newvisions}. 
This seems fundamental to the rules that are abstractly invoked and imposed in \emph{Metropolis} by established violence. 
The love between Cindi Mayweather  and Anthony Greendown is dangerous because (like interracial love and marriage), its existence would be a reflection of equality, of recognized personhood, when the established order's smooth functioning is contingent upon the denial of equality of personhood across racial (or android) lines. 
This is the exact same (racist) logic that fortified interracial marriage bans for so long (and interracial love, as how can we have one without the other?).
Mon\'ae's first choice in \emph{Metropolis} is to set the stage with one of the oldest battles for equality under the law: the right to love across racial lines.
In doing so, she places us squarely in the first wave of Black Feminism, situating her readers in a time and location where violence against black bodies is institutionalized and love is regulated.

% From the beginning, Mon\'ae centers violence against black women.
% Before she becomes ``An outlaw, out running the law" \cite{neonvalleystreet}, she is thrust into a world she had thought ``impossible!" \cite{happyhunting}, a world where the concept of her safety and agency is radical and revolutionary.
\emph{The Chase Suite} begins a conversation about first-wave black feminism in the terms that are fundamental to its aims: physical safety.
This is clear in the dance anthem off the first album, \emph{Many Moons}, which offers raw critiques without the nuance of theory or the kinetics of activism.
Cindi Mayweather laments her use as a sexualized humanoid, and her simultaneous lack of access to the more meaningful associated pursuit of love: ``A pretty face makes a pretty picture, but fall in love and they're coming to get you" \cite{happyhunting}.
She calls for her freedom, and bootstrap ideologies croon back, faulting her as the source of her enslavement: ``I keep my feet on solid ground. Freedom, Freedom, Freeeeedom! // \emph{Your free but in your mind, Your freedom's in a bind!}" \cite{manymoons} (emphasis mine to denote multiple voices).
An unfortunate feature of the first wave of black feminism was the consistent overlook of the voices of black women, from W.E.B Du Bois' lack of citationality for the brilliance of Anna Julia Cooper to Angela Davis' relative marginalization within the Black Panther Party.
Mayweather faces this struggle, and is given the means to amplify her voice (a microphone) only when a man grants it: ``let me just hear you squeak'' \cite{manymoons}.
Later in \emph{Metropolis}, she is no longer reliant on men to broadcast her message, a representation of Black Feminism as an emerging voice that is not a feature of dialogues but a separate mechanism for social change. 
%She is eventually offered shelter by someone who understands that the injustices she faces are a feature of the world, not her situation, a role that black women have to play for one another when their narratives are marginalized and cast aside by positivist/objective frameworks:
%``And when the world just treats you wrong // just come with me and I'll take you home" \cite{manymoons}.

Mon\'ae continues her tour through the first wave of black feminism by citing the Baptist tradition of respectability politics described in our course by Higginbotham \cite{higginbotham}.
In \emph{Sincerely, Jane.}, Mayweather recounts her mother's lamentations of how their world is crumbling, how the choices of her neighbors, along with cycles of poverty, incarceration and violence have crippled their community:
``These kids round' killin each other, they lost they minds, they gone. They quittin' school, making babies and can barely read. Some gone off to their fall, lord have mercy on them" \cite{sincerelyjane}. 
Unique to \emph{Sincerely, Jane.} is the invoking of religious authority as the mechanism for reasoning about injustice and seeking the amelioration for its ills, a feature of its voice, which is distinctly \emph{not} Cindi's, but that of her mother.

Mon\'ae uses Cindi's mother to embody central themes of first wave black feminism.
She believes that black women must take ownership for the destruction of the vices which are used to justify the oppression of black communities. 
While the majority of \emph{Metropolis} focuses on the ways that justice is \emph{structurally} denied to oppressed communities (both android and black) \cite{queen}, \emph{Sincerely, Jane.} recognizes and airs out this argument without advocating for it.
Today, this argument is thought to paralell the lines of the Moynihan report: insufficient and misdirected as a mechanism for the resolution of systematic injustice. 
Thus, rather than using her own voice, Mayweather is quoting her mother, simultaneously giving deference, citationality and respect to older generations of Black Feminists, while noting their differences in theory and perspective.

A re-release of \emph{The Chase Suite} offered two more ballads, which are simple lyrically and theoretically, but are powerful and moving in their sparsity and beauty.
In both \emph{Smile} and \emph{Mr. President}, Mon\'ae's voice takes on its personal, soulful, operatic accuracy, which over a comparably sparse background, beams with clarity, simplicity and truth.

\emph{Mr. President} gives \emph{Metropolis} its first overtly political demands, questioning capitalist mentalities and interventionist foreign policy: 
``We can't go fighting wars with hearts of hatred, our nation's greed won't make it better, or quiet the fear in our hearts" \cite{mrpresident}.
Beyond these clear claims, \emph{Mr. President} asks for leaders to directly think about their marginalized constituents, and questions why policies so rarely reflect the needs of its aspiring citizens who seek to escape poverty:
``Fuel is running low, and I've got places to go" \cite{mrpresident}.

Complimentarily, in \emph{Smile}, Mon\'ae (or perhaps Mayweather) offers us a beautiful, crushing, view into the culture of dissemblance described by Darlene Clark Hine \cite{hine}.
``Smile, though your heart is aching // Smile, even though it's breaking" \cite{smile}.
Though not mentioning the specific reasons that Black Women must smile to save their lives, Mon\'ae spells out dissemblance like a mother would to her young daughter; as hope that a smile might protect her daughter from the pain she knows the world will invent, and the hope that a smile might lead her to a brighter future:
``Smile, through all fear and sorrow // Smile and may be tomorrow // You'll find the sun come shining through // For you" \cite{smile}.

The piece of work that first got me thinking about the Black Woman's Standpoint was \emph{for colored girls who have considered suicide / when the rainbow is enuf}.
In the scene `i'm a poet who', the lady in orange says that ``we gotta dance to keep form cryin and dyin" \cite{forcoloredgirls}, another use of external joy as a mask and a relief from tragedy.
Dissemblance might have been coined by Clark Hine as a response to sexual violence, but it is so clearly a reaction to injustice, in spaces big and small, in situations annoying and life threatening.
Mon\'ae (Mayweather) offers a simple, uni-directional compass to navigate a world that doesn't value the lives of Black Women:
``You must keep on trying // Smile, whats the use of crying // You'll find that life is still worthwhile // If you just smile" \cite{smile}.
Dissemblance, performance and secrecy are all knit closely together, and though we first get the language to discuss it in the third wave, its roots are clearly a feature of the earliest struggles that Black Women have overcome.
No wonder Mayweather later notes to her lover ``I know I'm mysterious sometimes, and you are mysterious too" \cite{primetime},
mystery and secrecy have saved her life.

%------------------------------------------------

\section*{The ArchAndroid Suite II: Performance, Sanity and Black Women in Activism}

If suite one was about escaping physical violence, suite two is about moving beyond violence, looking to secure opportunity, solidarity, and power for black women.
This transitions us into the second wave of black feminism, with two major themes: galvanizing the support of black women as a tool for their prosperity, and the challenges that black women face in their roles in activism.

In \emph{Dance or Die}, Cindi Mayweather is getting her fellow female androids excited about activism in common sense terms:
``When you get elevated, they love it or they hate it... They tryin' to take all of your dreams but you can't allow it" \cite{tightrope}.
In inspiring android women to activism, she acknowledges that hidden \emph{systems} of oppression are responsible for their shared condition, not purely the people who are the \emph{faces} of those systems:
``Just keep rebelling away, you gotta dream it away // Because the weatherman on TV ain't creating the weather" \cite{danceordie}. 
In the second suite we see deep engagement with ideas of the second wave of Black Feminism: that \emph{de jure} rights do not guarantee freedom, that the lives of black women are still at risk, despite the fact they might have more access to more establishment protections.
Additionally, we see the hardening of ideological lines (much as we do within the second wave of Black Feminism), bystanders are no longer neutral, Mayweather implores to her audience that silence is political:
``This is a cold war, do you know who you're fighting for" \cite{coldwar}.

Additionally, Mon\'ae uses the space of the second suite to question how men and women in the black and android communities struggle differently for their liberation, and ask what that liberation looks like for each.
In doing so, Mon\'ae is trying to separate out the quest for liberation from oppression from the quest for socially divined power, and questions the conventional wisdom that equates the success of members of a race with the success of its male identified members:
``This man wants to see another jubilation //
And that man wants us in emancipation //
And then there is the man who wants a stronger nation //
You see we really got to and I think that we ought to //
Protect the mind from degradation //
Sow in the seeds of education //
They run from us, are we that dangerous? //
There's a war in all the streets and yes the freaks must dance or die!" \cite{danceordie}.

In drawing a contrast between the aims and methods of men and women androids, Mon\'ae sets Mayweather up to be the leader of the revolution she has started, and provides justification and response for that leadership.
``Call me weak, or better yet you can call me // you can call me your hero, baby" \cite{faster}.
This reminds me of Anna Julia Cooper's 1892 remarks about the spaces that Black women inhabit:
``Only the black woman can say `when and where I enter, in the quiet, undisputed dignity of my womanhood, without violence and without suing or special patronage, then and there the whole...race enters with me'" \cite{cooper1892}.
Mayweather casts herself as the hero of the android community because her liberation will truly mean the liberation of all androids.
When the weak are heroes, presumably all strong.

The second half of the second suite is characterized by songs about internal and external sanity and peace.
In \emph{Oh Maker}, Mayweather talks to her maker (god? human?) about love, beauty and loss, and the toll it exacts on her \cite{ohmaker}.
In \emph{Cold War}, Mayweather laments her consciousness in a world where her self (and self actualization) is systematically suppressed:
``I'm trying to find my peace.  I was made to believe there's something wrong with me! And it hurts my heart, lord have mercy, ain't it plain to see?" \cite{coldwar}. 

In \emph{Come Alive (War of the Roses)}, Mon\'ae paints a grim picture of Mayweather's perceived mental illness, through the opinion of others.
Though Mayweather doesn't question her own sanity, others do.  
They say she is a mess, that she needs to take her meds, but Mayweather just describes herself as `coming alive' \cite{roses}.
In this, I think there is a critique of the way that we validate (or systematically ignore) the black woman's standpoint, particularly around activism, agitation and anger.
In \emph{Metropolis}, society questions the mental health of the radical, not the social health of the system she has `come alive' to fight against.

Finally, an important discussion about abusive relationships is started in the second suite, but left deliberately open, for resolution in the third:
``I'm locked inside a land called foolish pride // where the man is always right // he hates to talk but loves to fight // is that alright?" \cite{lockedinside}
\emph{Locked Inside} is wonderful in that it doesn't trivialize domestic iolence. 
It doesn't infantilize the standpoint of women in similar situations, it doesn't prescriptively reject the situation, nor does it cast judgement upon it, rather, it discusses it.
That conversation becomes the focus of the third suite, which names and attempts to resolve the question that Mayweather has begun to ask herself:
What are the limits to her love?

%------------------------------------------------

\section*{The ArchAndroid Suite III: Naming Love and its Limits}

Cindi Mayweather is awakened to her world by the consequences of discovering and creating love.
In the third suite of \emph{Metropolis}, Mayweather grapples with her love for Anthony Greendown in moving ballads, careful considerations and unspoken understandings.
Though Mayweather has sacrificed for their relationship, and knows of its power and beauty, most of the third suite is about grappling with the limitations of that love.
Mayweather finds that the love which has singled her out, which has awoken her, \emph{is not what defines her}, she finds that it cannot give her the freedom that she is compelled to.

Mon\'ae begins the suite with a discussion about sex, both in terms of sexual awakening and sexual deviancy.
In \emph{Wondaland}, Mon\'ae's singsong narrator implores: ``Take her back to wondaland, she thinks she left her underpants" \cite{wondaland}.
In \emph{Make The Bus}, Mayweather is confessing that she has ideas and temptations that she feels she ought avoid:
``I've got this terrible fixation // ... // just want to keep it in the realm of fantasy" \cite{makethebus}.
By starting with open and honest, unapologetic discussions on sex (``This is your life. This is my life. We belong here, stay the night" \cite{wondaland}) 
Mon\'ae centers sex as both a natural part of a young black woman's journey, and sets the stage to discuss its misuse.

In \emph{Neon Valley Street} (my unequivocal favorite of any of Mon\'ae's works, by the way), Mayweather reminisces on her and Greendown's past (on the street where she had lived before her fugitive life), and the elaborate, intricate, careful detail she has placed in to her affection and care reverberates through her voice into hearts, minds and circuits.
``May this song reach your heart // may your ears love the sweet melody // Every note, every chord // I've arranged it for you and for me'' \cite{neonvalleystreet}.
Love's rarity, beauty and intimacy is on display in the third suite, but attached to it are questions of what love can look like, and how one performs it.
In \emph{Say You'll Go}, Mayweather implores her partner to be willing to pursue love with the knowledge of its challenges and scarcity:
``Love is such a novelty // A rarely painted masterpiece // A place few people go or ever know" \cite{sayyouwillgo}.

But Mayweather knows that in asking her partner to `say he'll go', there is a power question associated, one of ``Who will lead? Who will follow?''
 \cite{sayyouwillgo}, echoing bell hooks' language on power, love, and trust discussed above in \emph{New Visions} \cite{newvisions}.
Power and love are frequently at play in \emph{Metropolis}, and from what little we have of their relationship, it seems as if Mayweather is calling the shots, while Greendown has the agency to execute on them because of his relative power as a human: ``Sir Greendown // Let's leave in an hour // Meet me at the tower // Ride your horse" \cite{greendown}.
This power imbalance is always at the heart of her questions on love, and hooks, as mentioned before, questions whether love can ever truly exist as a process between people with any sort of power over one another \cite{newvisions}.

\emph{All About Love: New Visions} also presents us with a question about how Mayweather (and Mon\'ae) perform and describe love.
Throughout the five suites of \emph{Metropolis}, Mon\'ae uses the word love as hooks would, as a verb (though ignoring hooks' call for its sparing use).
Hooks impresses upon her readers the importance of viewing love as an active process and series of states of being, not a commodity `like dixie cups' which can be attained and lost:
``How different things might be if, rather than saying `I think I'm in love,' we were saying `I've connected with someone in a way that makes me think I'm on the way to knowing love.' Or if instead of saying `I am in love' we say `I am loving' or `I will love'. Our patterns around romantic love are unlikely to change if we do not change our language.'' \cite{newvisions}.
Similarly, Mon\'ae is careful to avoid using love as an object, treating it alternatingly as a verb and as an objective: ``Tell me are you bold enough to reach for love?'' \cite{manymoons}.

The penultimate track of the third suite is the rich and long 57821, named after Cindi Mayweather's system assigned serial ID number.
In the Deep Cotton featured song, there is storytelling craft buried beneath a choral masterpiece, describing Cindi's imprisonment, her longing for her beloved, and a carefully wrought connection between the love that she has developed, built and felt longing for, and an idea of her deification.
In the song, there is the introduction of explicit language around `the One', or `the ArchAndroid', a figure we later learn is a feature of android mythology: a revolutionary who will bring humans and androids together as one.
This idea of reunification is intertwined with Mayweather's captivity and longing for her beloved: is she the one who can unify humans and androids? Is she the one who can create and maintain love with Anthony?

The second of these questions is answered in the final song of the suite, BabopbyeYa, where we get the allusion to domestic abuse as a feature of the relationship between Mayweather and Greendown.
Mayweather spends the first three quarters of the song (which reads like a broadway power ballad) describing her layers of infatuation, admiration and pleasure that she gets from Greendown.
However, the song crecendos to a peak and immediately simmers to a cool, where Mayweather offers us the only hint to darker tones of their relationship: ``This time I shall be unafraid // And violence will not move me" \cite{babopbyeya}.
Having been trained in recognizing patterns of partner abuse, this line deeply concerns me.
On again, off again relationships with mutual co-dependance (`this time'), and a history of violent interactions (`violence will not move me') are both red flags for partner abuse predicated on power imbalances.
With this one line, Mon\'ae shifts the picture, alluding to the existence of domestic, and simultaneously acting as example in how difficult it can be to name and discuss domestic violence openly, particularly in marginalized communities.

The question of whether or not to discuss domestic violence within the Black community was a debate of the second wave of black feminism.
Many were perturbed to see Toni Morrison including rape and incest in \emph{The Bluest Eye}, which many criticized for its role in degrading the `strength' of the black community.
This shows the flaws in the the thinking of the time, and the way in which Black Women's narratives and freedom were put as tertiary objectives behind those of black men and white women.
The tangible break between the second and third wave of black feminism seems to lie in this space, in the Anita Hill `inquisition', in this question of prioritization, self advocacy, and sexual violence.

In \emph{Outlaw Culture: Resisting Representations}, bell hooks gives us the language to describe how domestic abuse as a mechanism of power is perpetuated by the `seduction of violence', from which the tenth chapter gains its name \cite{hooks1994outlaw}.
Hooks shows us how because Black men are systematically denied access to power in a white society, that sexual violence is the manifestation of power that is most available to them, and this garners and support a culture of rape and partner abuse.
Hooks discusses how the `seduction' of any power gained through the exploitation of others needs to be rejected to move forward, and that women can change this pattern by refusing to be seduced by power, and specifically, violence.


In Cindi's words, she has made a deliberate choice: she will no longer be moved (or seduced) through violence. 
Without this feature of her life, she sees ``beyond tomorrow // This life of strife and sorrow // My freedom calls and I must go" \cite{babopbyeya}.
Cindi Mayweather chooses her freedom from an abusive relationship over the pleasure that she acknowledges the relationship has brought her.
In \emph{New Visions}, hooks paints this decision within the context of love:``All too often women believe it is a sign of commitment, an expression of love, to endure unkindness or cruelty, to forgive and forget. In actuality, when we love rightly we know that the healthy, loving response to cruelty and abuse is putting ourselves out of harm's way'' \cite{newvisions}.

Just like the Anita Hill testimony awoke the Third Wave of Black Feminism in the form of Rebecca Walker \cite{rebeccawalker} (a pun on her claim of embodiment, not a singular citation), Mayweather's decision to leave Anthony is an inflection point in \emph{Metropolis}.
That choice launches her into the fourth suite: though she is no longer fighting for her own love, she has been awakened to fight for so much more. 
She has transitioned into the third wave of Black Feminism.


%------------------------------------------------

\section*{The Electric Lady Suite IV: Queerness, Performance and Activism}

It is tough to write about the fourth suite of \emph{Metropolis} without singularly centering the Black Feminist Anthem Q.U.E.E.N, but we will try.
In the fourth suite, Mon\'ae articulates a clear set of guidelines for activism, implores her audience to act against oppression, and begins honest conversations about queerness.
In doing so, Mon\'ae articulates a carefully hedged politic of radicalism: one which centers systematic injustices, rejects violence, and uses dance and song as a means of communication and resistance.

The most explicit manifestation of these themes are little `interludes', scattered throughout the fourth and fifth suites.
Each is a 1-2 minute excerpt from a radio station, WDRD (a thinly veiled Droid reference), complete with a DJ (Crash Crash), advertisements and a call in hotline.
The first interlude has two call ins, one from an android woman who is living Cindi's vision of resistance ``break some rules, just like Cindi'', the second of whom is a male droid (Bop Bot Say-What), who plans on ``hittin' someone in the head'' \cite{goodmorningmidnight}.
DJ Crash Crash makes it clear that the first vision of radicalism is great, but challenges the second as some ``rusty dusty, nano-thinkin' nonsense'' \cite{goodmorningmidnight}.

The fourth suite gives us the language to describe why violence is so appealing, and why non-violence is so hard to sustain:
``I'm praying for the man sitting without much time // May he understand the clock will never rewind // Wisen him and sharpen him and give him a motto // hate no more, said he must hate no more" \cite{danceordie}.
Violence is seductive for those without other means of redress, without those with education, without motto.
Throughout \emph{Metropolis}, Mon\'ae carefully draws this line between anger, activism and violence.
Nonviolence is a choice, and a strategic choice, one that Mayweather has made for herself  but also for her movement: ``I use my words when stones come around'' \cite{manymoons}.
In an interlude called Good Morning Midnight (a choice which alludes to a 1939 Novel which grapples with the feelings of depression, lonliness and vulnerablility felt in between the world wars, likely as an allusion to the physical, mental and emotional tolls that activism takes), Mayweather's followers at the radio station WDRD have a caller, `Bop Bot Say What' who proposes violence as a means to redress the clear inconsistencies of ruling logic: ``hitting us all up in the head and wondering why we don't think straight" \cite{chromeshoppe}.
Mayweather's followers quickly silence this voice, characterising him as a fool to be avoided. 
It should be clear from these examples that the fourth suite is really comprised of Mayweather carefully selecting, pruning and evolving her radical politic, as the android community begins to look at her more and more as `the one', rather than one of their own.


One of the features of this politic of activism is a consistent call to universal mobilization, but distinctly not universal oppression as a grounds for it.
Mayweather is singing to her community, but invites in the 
``Sudanese and Congolese who put the roll in the rock // From here to Sudan, Metropolis to Iraq" \cite{danceordie}.
Mayweather demands that apathy is not acceptable, in questioning us each to think ``who we are fighting for'' in \emph{Cold War} \cite{coldwar}.
Moreover, Mayweather makes clear that ability should be the only thing getting in the way of activism: 
``March to the streets 'cause I'm willing and I'm able" \cite{queen}.

Another central point in this articulation of activism is her unapologetic centering of her own voice and those of other black women and black activists icons.
In the \emph{Chrome Shoppe} Interlude, one of the female androids is named `Assatta' \cite{chromeshoppe}, in Q.U.E.E.N., Mayweather speaks to her aspirations to ``Keep leading like a young Harriet Tubman" \cite{queen}.
In one of the pivotal moments in the fourth suite, she asks the black woman to ``Wake up Mary" \cite{sallyride}, in a song titled after Sally Ride, with the impact of that awakening clear yet ambiguous: ``some amazing news // you've got the right to choose" \cite{sallyride}. Awakening was a key theme of the first suite (``Day dreamers please wake up, we can't sleep no more." \cite{sincerelyjane}) and it is revived in the form of a request, now as a choice rather than an imperative, as Mayweather begins her revolution: ``Electric Ladies will you sleep? Or will you preach?" \cite{queen}.

Additionally, Mayweather's politic of change centers around her music.
In doing this, Mon\'ae is making clear that her own work (in constructing metropolis) is a work of activism, artistic expression employed as a means to power and voice.
In cultivating an image for herself, Mayweather is redefining the way that she is viewed by other androids (as a strong leader with the aim of unification), and the way that she is viewed by humans (forcing them to grapple with her humanity).
This intentional image making reflect's hooks discussion on media as a form of power: ``...no matter how sophisticated our strategies of critique and intervention, [we] are usually seduced, at least for a time, by the images we see on the screen. They have power over us, and we have no power over them'' \cite{hooksreeltoreal}.
By reclaiming and establishing herself as media, Mayweather (and Mon\'ae) is reversing this script.
They are reclaiming (through their artistry) control over their image, and thus reclaiming fuller control of their selves.

Okay I give in, now we get to talk about Q.U.E.E.N. without further apology.
In the fourth suite, Mayweather is over Greendown, and is examining her own sexual preferences.
``Is it weird to like the way she wears her tights?" \cite{queen}
What I really like about Q.U.E.E.N. is that it isn't a coming out story.
It is a story about self exploration, about self knowledge. 
Q.U.E.E.N. isn't about preferring the company of women or men, it is about realizing that the very question is one that Mayweather is entitled to ask and decide for herself.

Moreover, that entire examination takes place within the context of her race: Mayweather's thoughts of female sexuality are established within her experience of black femininity.
She asks whether it is ``peculiar that she twerk in the mirror'', or ``is it rude, to wear, my shade'', and ``am I a freak because I love watching mary'', all of which are questions of sexual exploration explicitly within her personal and social location.
This evokes E. Patrick Johnson's discussion of his contextualization of his homosexuality within the context of his upbringing and culture in his work \emph{``Quare'' studies, or (almost) everything I know about queer studies I learned from my grandmother} \cite{johnson2001quare}.

After one more point, I will leave Q.U.E.E.N. to rest.
In the center of Q.U.E.E.N., there is a quote which summarizes the pivotal shift for LGBTQ rights (and racial language which describes it).
One of the weaker arguments for LGBTQ rights is the `born this way' argument: the idea that people ought give LGBTQ individuals rights and basic human dignity because they do not have the liberty or ability to choose their lifestyles.
This is a weak argument precisely because it undercuts what should be the central (and only?) argument for LGBTQ rights.
The correct argument is simple: LGBTQ individuals deserve respect and rights not because you happen to be related to one of them, not because they don't have a choice, not because they were born that way, but simply because they are people.
The idea that personhood should be accompanied by respected autonomy should not be radical.
Distinguishing between personal preference and genetic chance (or programming) is a compelling argument to people new to the concept of LGBTQ rights, but it undercuts the more important long term goal: a world without arbitrary sexual and social standards of normalcy, a world where birth as a person comes with the most fundamental of human pursuits: free autonomy. 
In one line, Mon\'ae succinctly summarizes what it has taken me years to understand: That our control over our sexual preferences, the ability to self reprogramming, shouldn't depend on an initial programming in the first place:
``Would your god accept me in my black and white? Would he approve the way I'm made? Or should I reprogram reprogram and get down?!?" \cite{queen}


%------------------------------------------------

\section*{The Electric Lady Suite V: Dissemblance, Self Empowerment and Victory}

The fifth suite is bittersweet.
Mon\'ae draws strength from her mother's toil toward a better life; she describes the necessity for joy in the sorrow of activism.
At the heart of the fifth suite, there is a discussion of heroes, and the strength of resilience in the face of struggle.
It draws from the questions that current day Black Feminism is discussing: the struggle of struggling, the centrality of individual empowerment within a cultural context of exemplars, and the necessity for self preservation in systems which will not care for you, for you.
Language from the recent history of Black Feminism richly describes the fifth suite, which celebrates small victories and acknowledges a long road.

It is disrespectful to paraphrase from \emph{Ghetto Woman}, the third track in the fifth suite of Metropolis, as Mon\'ae's language is too rich to sample from without doing some line a disservice, but we will try.
In it Mon\'ae (or Mayweather, it is not clear) thanks her mother for the work and love that she has given her.
Mon\'ae draws a long and connected arc, showing the poor Black Woman's otherness and lack of resources within a society which refuses to acknowledge (and certainly not attempt to understand) her, is somehow met with unbelievable strength and courage:
``She'd keep it to herself and nobody could understand her // Even when she thought that she couldn't she carried on // She couldn't imagine both of her daughters here all alone" \cite{ghettowoman}.
In that exact frame of her mother's isolation and struggle, Mon\'ae highlights her resilience, compassion and wisdom.

Moreover, Mon\'ae cites the exact cause of this mischaracterization of her mother:
``I wish they could just realize // That all you've ever needed was someone to free your mind // Carry on Ghetto Woman // Cause even in your darkest hours I still see your light" \cite{ghettowoman}.
By vigorously defending the light of the `ghetto woman', Mon\'ae (and Mayweather) are articulating a clear black feminist politic: that the power and strength of the most marginalized is immense, and their liberation is critical. 
This idea of the empowerment and liberation of the most marginalized is in apparent conflict with Mayweather's singular deification, a conflict which is resolved in the last three songs of the suite (and the work).

An interesting contradiction in Metropolis is Mayweather's singularity as `the One', with her politic of universal empowerment.
From the beginning of Metropolis, Mon\'ae makes it clear that the individual is the key to winning her struggle.
``The silver bullet's in your hand and the war's heating up" \cite{manymoons}.
This is at odds with the language of deification that Mayweather undergoes in her ascension to leadership among the Android resistance.
 If Mayweather is able to win her freedom because she is `the one', what does that mean for `the rest'?
This question isn't explicitly answered in the fifth suite, but I think that there are a few clues that point toward a hypothesis: Mayweather is not `the one', leaving room for the empowerment of other android women.

Evidence for this hypothesis comes from \emph{Sally Ride}, where Mayweather undergoes a personal crisis that makes her question her powers and strength:
 ``Take me to the river, my soul is looking for a word from God // Oh God, like a rose in the cold // will I rise?" \cite{sallyride}
Mayweather feels she needs to distance herself from her work, to find internal peace and move on from her work:
 ``I know you love me but I'm still gone. I've got to make my peace, I've got to move on''\cite{sallyride}.
 If Mayweather isn't `the One', it paints the last three songs on the album within the context of modern day Black Feminism: \emph{Victory}, \emph{Dorothy Dandridge Eyes} and \emph{What an Experience}.
 
 In \emph{Victory}, Mayweather comes to the realization that ``to be victorious, you must find glory in the little things''\cite{victory}.
This is a reflection of how draining and tiresome activism is; to reach the finish line you need to celebrate every step.
Mayweather, upon learning that she is not the one, realizes that she has still contributed toward the collective goal, that even though she is not a deity, her strength and perserverance have made a huge impact, and she views this revelation within the context of self growth: 
 ``Cause when the rain falls // My seed will grow // I'll be further to my dreams tomorrow" \cite{victory}.
 This sentiment about black feminist activism is beautifully summed up by Barbara Smith in the 1999 introduction to her anthology \emph{Home Girls}:
 ``Although the black feminist movement is not where I envisioned it might be during those first exciting days, it is obvious that our work has made a difference. Radical political change more often happens by increments than through dramatically swift events. Indeed, dramatic changes are made possible by the daily, unpublicized work of countless activists working on the ground" \cite{smithhomegirls}.
 Mayweather comes to the realization that her work is critical, even if change is so much slower than she would hope, even if she cannot take responsibility for changing all of it.
 
 In \emph{Dorothy Dandridge Eyes}, the idea that Mayweather is not the one is drawn out by attention to the things which make her unique and special \cite{dandridgeeyes}.
 In celebrating the things that make her unique, Mayweather is distancing herself from deification, from unification, from universalism.
 In \emph{What an Experience}, Mayweather (and to a degree Mon\'ae), reflect back on their work, appreciative of the experience that they have been through, the people they have met and loved, and the growth they have experienced. 
 These final songs seem to be addressing the emerging issues of resilience, victory, struggle, empowerment and energy within activism.

%Non-reliance on Dieties:
%``Indivisible Sum // Here's the book, now the saga's begun" \cite{fiveseveneighttwoone}
%``A long long way to find the one // We'll keep on dancing till she comes // These dreams are forever" \cite{danceordie}
%``I want to know if you believe Cindi Mayweather is not only just electric lady, number one and all, but also the ArchAndroid?" \cite{favoritefugitive}
%``Heaven is betting on us." \cite{primetime}

%Ending on a celebration of her unique features \cite{dandridgeeyes}, and a party \cite{whatanexperience}

%------------------------------------------------

\section*{Meta Analysis: Self Production and Transcending Categories}

Though examination of Metropolis has given us a bountiful amount to work with, there are elements of Mon\'ae's work that ought be examined and celebrated on a broader scale, elements that tie directly into Black Feminist work.

One element of Mon\'ae's work that should not be overlooked is her degree of control over her sound and image, within a capital-audio system that is hedged against that kind of freedom.
Mon\'ae discusses within her music how self-production and ownership of that process is a critical form of power: 
``She who writes the movie owns the script and the sequel" \cite{queen}.
In her 1986 paper \emph{Learning from the Outsider Within: The Sociological Significance of Black Feminist Thought}, Collins describes how Black Women (in academia, as well as intellectuals more broadly) have used their unique position as `outsiders within' to shape sociological theory \cite{collins86}.
One of the three components of Collins' analysis is how Black women have used their standpoint to ``challenge the political knowledge-validation process'' (Self-Definition), and ``replace externally-derived images with authentic Black female images'' (Self-Valuation) \cite{collins86}.
Though not an academic, Mon\'ae as an artist possesses distinctive forms of intellectual power and cultural sway
The very reality of having an audience, yet alone a self-producing label, gives Mon\'ae the artistic and intellectual freedom to make music the way she chooses, to tell the stories that she wants to tell in the way she wants to tell them.
Mon\'ae's self production means that the image she portrays is honest and is not warped by the perceptions and desires of capitalistic bias toward crowd pleasing and universal audiences.
This means that Mon\'ae's work is uniquely situated: it does attempt to achieve mainstream success and artistic prominence (the Grammys, etc), but does so without shedding its marginality.
When was the last time you heard a top-40 song about the plight of women living in ghettos?
This space that Mon\'ae inhabits, as a celebrated music icon and a proud Kansas City native is a rare one that was described to bell hooks by her mother in sending her off to college.
In encouraging her daughter to not let her resistance get in the way of her personal success, hooks describes her mother's warning as ``reminding me of the necessity of opposition and simultaneously encouraging me not to lose that radical perspective shaped and formed by marginality'' \cite{hookschoosing}.

A second element that runs throughout Mon\'ae's work catalogues the ills of categorization, and the limitations and harms of collectives.
This is most visible to me in Mon\'ae's frequent collaborations with an Atlanta based group Jazz electronica group, Jaspects.
Their collaborations, including \emph{My First Love}, \emph{Peachtree Blues} and \emph{2012}, feature Mon\'ae's stunning voice over the band's signature modern jazz sound \cite{peachtreeblues} \cite{myfirstlove}.
In her 2009 collaboration with the group, Mon\'ae's song 2012 implored listeners to ask what freedom means to them, and how they actualize it for themselves and for others.
Though she is not featured on the title song of the album ``The Polkadotted Stripe", freedom within the context of the album revolves around the idea of false dichotomies.
``We see black or white, polkadots or stripes. Freedom we deny, lets open our eyes" \cite{polkadottedstripe}.
Mon\'ae questions dichotomies and categorization throughout the Metropolis suite, through explicit and subtle means.
Though not always as clear as in Q.U.E.E.N. (``Categorize me, I defy Every Label" \cite{queen}), 
On her own style, Mon\'ae refuses to adhere to categorization which undermines the potency of her work:
``We're growing up in the iPod generation in terms of genres. I do away with labels and categories. I don't believe in them ... That was created by man." \cite{joeyguerra2010}.
This comes through in her musical and aesthetic choices.
Metropolis' style cannot be summed up succinctly, but there are elements of a plethora of musical styles and artistic modes, the inconsistency of which doesn't come accross as discoherence but as intention and mastery.
As always, Patricia Hill Collins gives us beautiful language to describe Mon\'ae's systematic rejection of the labels which follow her, and connects this back to the criticality of intersectional frameworks in the pursuit of equality: ``Additive models of oppression are firmly rooted in the either/or dichotomous thinking of Eurocentric, masculinist thought. One must be either Black or white in such thought systems--persons of ambiguous racial and ethnic identity constantly battle with questions such as `what are your, anyway?' This emphasis on quantification and categorization occurs in conjunction with the belief that either/or categories must be ranked. The search for certainty of this sort requires that one side of a dichotomy be privileged while its other is denigrated. Privilege becomes defined in relation to its other" \cite{collins2009}.

%------------------------------------------------

\section*{Conclusion}

Mon\'ae gives us so much language to process and understand Black Feminism that the uniqueness of \emph{Metropolis} is easily lost among its stunning components.
Metropolis examines situations that are mundane and spectacular, unjust systems and unabashed love with the consistent analytical lens of Black Feminism.
Just as we see Black Feminism change in form and nature over time, we see the same movement in the topics and trends of Metropolis.
Mon\'ae cites past schools of thought, examines oppositional narratives about Black Women, and establishes a carefully hedged politic of general pursuit of liberation, abstracted from specific agenda items. 
Mon\'ae translates academic language about matrices of domination into common sense (and catchy) lyrics which galvanize their repetition and prominence: ``Add us to equations but they'll never make us equal" \cite{queen}. 

Mon\'ae's work is of critical cultural importance because, just like Mayweather uses her voice for the empowerment of android women, Mon\'ae's work centers the empowerment of young black women in a way that does not exceptionalize herself, rejecting the too frequent American narrative of exceptionalism and singularity. 
Mon\'ae's work is critical because it empowers young black women by celebrating their unique selves, while refusing to allow them to be defined by their categories. 
Metropolis is a road map, but also a survival guide, helping young black women think about how they can navigate a world that actively seeks their destruction.
In their 2002 work on how girls of color can be intentionally educated and counseled to thrive in systems which frequently invalidate their experiences and voice, Iglacias and Cormer discuss the adolescent transition as one that is rooted in struggle, and that ``A girl who will develop as a woman who embraces freedom has a strong sense of self-confidence and self-efficacy'' \cite{iglesiascormier}.
It is in honestly portraying the realities of living as a Black Woman (love, loss, joy, injustice, activism and energy) that Mon\'ae prepares a younger generation to successfully navigate the spaces that she has been able to navigate so successfully.
Mon\'ae asks her listeners to find themselves as Electric Ladies, to wake up from their sleep.
They will awaken in a status quo that systematically unjust, where they are denied voice and credibility, but one where they can be their become their own heroes.



%------------------------------------------------


\section*{Misc/Unassigned}

BOOK IDEAS:

"The true focus of revolutionary change is never merely the oppressive situations which we seek to escape, but that piece of the oppressor which is planted deep within each of us." \cite{lordredefiningdifference} 

``What lies between the lines are the things that women of color do not tell each other. There are reasons for our silences: the change in generation between mother and daughter, the language barriers between us, our sexual identity, the educational opportunities we had or missed, the specific cultural history of our race, the physical conditions of our bodies and the labor... We begin by speaking directly to the deaths and disappointments. Here we begin to fill in the spaces of silence between us. For it is between these seemingly irreconcilable lines--the class lines, the politically correct lines, the daily lines we run down to each other to keep difference and desire at a distance--that the truth of our connection lies." \cite{thebridge}

``As women we have been taught to either ignore our differences or to view them as causes for separation and suspicion rather than as forces for change. Without community, there is no liberation, only the most vulnerable and temporary armistice between an individual and her oppression. But community must not mean a shedding of our differences, not the pathetic pretense that these differences do not exist." \cite{lorde2003master}




``If any female feels she need anything beyond herself to legitimate and validate her existence, she is already giving away her power to be self-defining, her agency.'' \cite{hooksforeveryone}




%----------------------------------------------------------------------------------------
%	BIBLIOGRAPHY
%----------------------------------------------------------------------------------------

%\nocite{*}
\bibliography{ArchAndroid,Metropolis,ElectricLady,ClassReading,External}{}
\bibliographystyle{unsrt}

%----------------------------------------------------------------------------------------

\end{document}