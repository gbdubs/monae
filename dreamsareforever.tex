\documentclass[a4paper, 11pt]{article} % Font size (can be 10pt, 11pt or 12pt) and paper size (remove a4paper for US letter paper)

\usepackage[protrusion=true,expansion=true]{microtype}

\usepackage[T1]{fontenc}
\linespread{1.05}

\makeatletter

\renewcommand{\maketitle}{ % Customize the title - do not edit title and author name here, see the TITLE block below
\begin{flushright} % Right align
	{\LARGE\@title} % Increase the font size of the title

	\vspace{40pt} % Some vertical space between the title and author name

	{\large\@author} % Author name
	\\\@date % Date

	\vspace{30pt} % Some vertical space between the author block and abstract
\end{flushright}
}

%----------------------------------------------------------------------------------------
%	TITLE
%----------------------------------------------------------------------------------------

\title{\linespread{2.1}\textbf{These Dreams are Forever}\\ % Title
Janelle Mon\'ae's Metropolis as a Historiography of Black Feminism} % Subtitle

\author{\textsc{Grady Ward} % Author
\\{\textit{Brandeis University}}} % Institution

\date{\today} % Date

%----------------------------------------------------------------------------------------

\begin{document}

\maketitle % Print the title section

%----------------------------------------------------------------------------------------
%	ABSTRACT
%----------------------------------------------------------------------------------------

\begin{abstract}

Janelle Mon\'ae's \emph{Metropolis Suite} offers its listeners a view into a dystopian world that marries elements of futurism with past and current conversations about oppression and freedom.
The analogy of black folk as androids is central, pervasive and interchangeable. 
Like the concept of enslavement, androids were `constructed' in the minds of their masters to serve capitalistic systems, and their masters are (over time) forced to come to terms with the fact that something much more human lies underneath their abstraction as machinery. 
Lyrically, stylistically, in the impassioned ballads and in the small details, Janelle Mon\'ae practices and educates her listenership about Black Feminism, showing how the liberation of the most oppressed creates routes for the liberation of all.
Moreover, Mon\'ae calls us all to action toward a better future with a carefully hedged politic of radicalism and self-empowerment.

\end{abstract}

\vspace{30pt}

%----------------------------------------------------------------------------------------
%	ESSAY BODY
%----------------------------------------------------------------------------------------

\section*{Introduction}

Janelle Mon\'ae caught my attention with her voice, she captured my mind with her lyrics, and she won my heart with her politics.
Mon\'ae is an American musician, mogul, actress and songwriter from Kansas City, Kansas.
The majority of her discography consists of \emph{Metropolis}, a three-album musical drama divided into five suites, which blends the lines between music, fable and politics.
With shoe-tapping dance hall anthems, operatic ballads, unapologetic rap, dissonant punk and shimmering orchestral overtures, the \emph{Metropolis Suite} cannot be surmised succinctly in style nor tone.
It follows the story of an android (Cindi Mayweather) who falls in love with a human (Anthony Greendown) and is sentenced to death for this crime \cite{wolfmasters}.
Cindi escapes her initial surprise to become a fugitive \cite{manymoons}, then an underground musical icon \cite{favoritefugitive}, then a society shaking radical \cite{fiveseveneighttwoone}, fighting for her freedom and the freedom of her people through myth and music.

Evoking images and ideas from Fritz Lang's Weimar classic by the same name, \emph{Metropolis} centers and shares the same themes of class, love and revolution \cite{metropolis}.
Additionally, though both have central imagery of mechanical women, Mon\'ae questions the the construction of women by their society: ``I'm a product of metal? I'm a product of the man." \cite{happyhunting}. 
Rather than having mechanical women as a means of controlling the proletariat, as in Lang's work \cite{metropolis}, Mon\'ae's Android heroine realizes her mission and capability to win the freedom of the oppressed.
Though both \emph{Metropolis} works center around protagonists learning about how capitalism is predicated on the exploitation of the opressed, their plots and lenses could not be more different.
Unlike in Lang's work, Mon\'ae centers the Woman as the revolutionary leader, rather than than filling that role by dispossessed aristocracy.
Unlike in Lang's work, Mon\'ae realizes freedom for her characters on their own terms, by their own means, without the need for establishment validation.
Unlike in Lang's work, Mon\'ae's heroine saves herself.

Though these contrasts are a starting place for a discussion of Mon\'ae's work, they mainly begin to shed light on how Mon\'ae's choices are painstakingly deliberate.
It can be difficult to consider the profound consequences that Mon\'ae's \emph{Metropolis} presents, often so buried underneath layers of distracting musical excellence.
But make no mistake: in \emph{Metropolis} Mon\'ae has chosen to rewrite a classic.
She is making a deliberate and strategic descision to center an othered woman as the focal point of social and radical change.
She is taking an established masterpiece of world cinema, and challenging not only its plot, but its epistemology.

The \emph{Metropolis Suite} is embodied Black Feminism.
Through fiction and musical fusion, Mon\'ae has the freedom and power to construct elaborate parables that are opaque and transparent, varried but consistent.
Moreover, by tracing and naming Black Feminist ideas and motifs through the three albums, we see an intentional and deliberate historiography of Black Feminism, through its three waves.
In this paper, we will discuss how Mon\'ae crafts a story which mirrors the history of Black Feminism, we will show how she uses the standpoint of an `ordinary' android to demonstrate the extraordinary potential of the oppressed individual, and we will enumerate how her music empowers and educates her listeners on Black Feminism without ever relying on theory.
Mon\'ae paints the Android a path to humanity, the dispossessed a path toward establishment, and the oppressed a path to freedom.
We will trace out these paths. 

%------------------------------------------------

\section*{The Chase Suite: Voice, Violence and Dissemblance}

``You know the rules!" \cite{wolfmasters}, the singsong establishment chides.
Reasoning is devoid from the call to violence, only adherence to rules governing love and humanity.
The Chase Suite, the first album and suite of \emph{Metropolis} sets the stage for Cindi Mayweather's awakening to a world not of her making.
Rather than her name, she is introduced by her serial identification number, 57821.
She has fallen `desperately in love' with a human, Anthony Greendown, a crime in the eyes of the state, and the gleeful narrator is announcing a chase for her capture and destruction.
Between the arcane and specific rules issued for her capture (`chainsaws and electro-daggers!' \cite{wolfmasters}), and the casual tone, the fugitive slave law implications are clear.
Less so are the ties to the pleasure that violence and suppression bring out in the ruling majority.
Postcards of lynchings are not a feature of the archaic past, and violence predicated on sexual identity and preference is still very much alive.

In bell hooks treatise on love, power and relationships \emph{All About Love: New Visions}, a message which resonated with me was the idea that love cannot exist within a power imbalance; that the fundamental nature of love as a verb, as a practice of care and intimacy, places as a prerequisite the full and equal humanity of its participants \cite{newvisions}. 
This seems fundamental to the rules that are abstractly invoked and imposed in \emph{Metropolis} by established violence. 
The love between Anthony Greendown and Cindi Mayweather is dangerous because (like interracial love and marriage), its existence would be a reflection of equality, of recognized personhood, when the established order's smooth functioning is contingent upon the denial of equality of personhood across racial (or android) lines. 

With a bang, Mon\'ae centers violence against black women, particularly from a system of disembodied rules that she does not have influence over or justification for.
Before she becomes ``An outlaw, out running the law" \cite{neonvalleystreet}, she is thrust into a world she had thought ``impossible!" \cite{happyhunting}, a world where the concept of her safety and agency is radical and revolutionary.
The Chase Suite begins a conversation about first-wave black feminism in the terms that are fundamental to its aims: physical safety.
Mon\'ae continues the conversations in the dance anthem off the first album, Many Moons, which offers raw critiques without the nuance of theory or the knetics of activism.
Cindi Mayweather laments her use as a sexualized humanoid, and her simultaneous lack of access to the more meaningful associated pursuit of love: ``A pretty face makes a pretty picture, but fall in love and they're coming to get you" \cite{happyhunting}.
She calls for her freedom, and bootstrap ideologies croon back, faulting her as the source of her enslavement: ``I keep my feet on solid ground. Freedom, Freedom, Freeeeedom! // Your free but in your mind, Your freedom's in a bind!" \cite{manymoons}.
She is eventually offered shelter by someone who understands the injustices she faces are a feature of the world, not her situation, a role that black women have to play for one another when they are dlegitimzed and cast aside by positivist/objective frameworks:
``And when the world just treats you wrong // just come with me and I'll take you home" \cite{manymoons}.

Mon\'ae continues her tour through the first wave of feminism by citing a scathing critique based in the Baptist tradition of respectability politics coined by Higginbotham \cite{higginbotham}.
In \emph{Sincerely, Jane.}, Mayweather recounts her mother's warnings of how their world is crumbling, how the choices of her neighbors, along with cycles of poverty, incarceration and violence have crippled their community.
``These kids round' killin each other, they lost they minds, they gone. They quittin' school, making babies and can barely read. Some gone off to their fall, lord have mercy on them." \cite{sincerelyjane}
Unique to \emph{Sincerely, Jane.} is the invoking of religious authority as the mechanism for reasoning about injustice and seeking the amelioration for its ills, a feature of its voice, which is distinctly \emph{not} Cindi's.
With this perspective, Mon\'ae highlights central themes of first wave black feminism, in that the black community (and primarily black women) must take ownership over the destruction of vice which was used to justify the oppression of black communities. 
While the majority of \emph{Metropolis} focuses on the ways that justice is structurally denied to communities of difference (both Android and Black), \emph{Sincerely, Jane.} recognizes and airs out this argument without advocating for it.
Rather than giving her own voice to this critique of her community (which in today's world is generally understood to be more along the lines of the Moynihan report, insufficient as a mechanism for the resolution of systematic injustice), Mayweather is quoting her mother, simultaneously giving deference, citationality and respect to older generations of Black Feminists, while noting their differences in theory and perspective.

A re-release of the \emph{Chase Suite} offered two more ballads, which are simple lyrically and theoretically, but are powerful and moving in their sparsity and beauty.
In both Smile and Mr. President, Mon\'ae's voice takes on its personal, soulful, operatic accuracy, which over a comparably sparse background, beams with clarity, simplicity and truth.

Mr. President gives \emph{Metropolis} its first overtly political demands, questioning capitalist mentalities, interventionist foreign policy: 
``We can't go fighting wars with hearts of hatred, our nation's greed won't make it better, or quiet the fear in our hearts" \cite{mrpresident},
but more than that, Mr. President asks for leaders to directly think about their constituents, and questions why policies so rarely reflect the needs of its aspiring citizens without financial resources:
``Fuel is running low, and I've got places to go" \cite{mrpresident}.

In Smile, Mon\'ae (or perhaps Mayweather) offers us a beautiful, crushing, view into the culture of dissemblance described by Darlene Clark Hine \cite{hine}.
``Smile, though your heart is aching // Smile, even though it's breaking" \cite{smile}.
Though not mentioning the specific reasons that Black Women must smile to save their lives, Mon\'ae spells out dissemblance like a mother would to her young daughter, as hope that a smile might protect her daughter from the pain she knows the world will create, and perhaps a smile might even lead her to a brighter future:
``Smile, through all fear and sorrow // Smile and may be tomorrow // You'll find the sun come shining through // For you." \cite{smile}
The piece of work that first got me thinking about the Black Woman's Standpoint was \emph{for colored girls who have considered suicide / when the rainbow is enuf}.
In the scene `i'm a poet who', the lady in orange says that ``we gotta dance to keep form cryin and dyin" \cite{forcoloredgirls}, a parallel that can't be ignored.
\emph{For colored girls} gives us the emotion to try to grapple with the challenge that is black womanhood.
Dissemblance might have been coined by Clark Hine as a response to sexual violence, but it is so clearly a reaction to injustice, in spaces big and small, in situations annoying and life threatening, Mon\'ae (Mayweather) offers a simple compass to navigate a world that doesn't value the lives of Black Women:
``You must keep on trying // Smile, whats the use of crying // You'll find that life is still worthwhile // If you just smile." \cite{smile}
Dissemblance, performance and secrecy are all tied together, and though we first get the language to discuss it in the third wave, its roots are clearly a feature of the earliest struggles that Black Women have overcome.
No wonder Mayweather notes to her lover ``I know I'm mysterious sometimes, and you are mysterious too." \cite{primetime}
Mystery and secrecy have saved her life.

%------------------------------------------------

\section*{The ArchAndroid Suite II: Performance and Women Owning Activism}

If suite one was about escaping physical violence, suite two is about moving beyond violence, looking to secure oporotunity, solidarity, and power for black Women.
In \emph{Dance or Die}, Cindi Mayweather is getting her fellow female androids excited about activism in common sense terms:
``When you get elevated, they love it or they hate it... They tryin' to take all of your dreams but you can't allow it." \cite{danceordie}
While simultaneously acknowledging that systems of oppression are responsible for their condition, not purely the people who are the faces of those systems:
``Just keep rebelling away, you gotta dream it away // Because the weatherman on TV ain't creating the weather." \cite{danceordie}
In the second suite we see deep engagement with ideas of the second wave of Black Feminism: that de jure rights do not guarantee freedom, that the lives of black women are still at risk, despite that they might have more access to more establishment protections.
Additionally, we see the hardening of ideological lines (much as we do within the second wave of Black Feminism), bystanders are no longer neutral, Mayweather implores that silence is political:
``This is a cold war, do you know who you're fighting for" \cite{coldwar}.

Additionally, Monae uses the space of the second suite to question how men and women in the black and android communities struggle differently for their liberation, and ask what that liberation looks like:
``This man wants to see another jubilation //
And that man wants us in emancipation //
And then there is the man who wants a stronger nation //
You see we really got to and i think that we ought to //
Protect the mind from degradation //
Sow in the seeds of education //
They run from us, are we that dangerous? //
There's a war in all the streets and yes the freaks must dance or die!" \cite{danceordie}
And in doing so, Monae sets Mayweather up to be the leader of the revolution she has started.
``Call me weak, or better yet you can call me // you can call me your hero, baby" \cite{faster}.
In this, Monae marks the importance of black female leadership in fights for equality, and reminds us all of Anna Julia Cooper's 1892 remarks about the spaces that Black women inhabit:
``Only the black woman can say `when and where I enter, in the quiet, undisputed dignity of my womanhood, without violence and without suing or special patronage, then and there the whole...race enters with me.'" \cite{cooper1892}.

Mayweather also recognizes the precariousness of her newfound activism, and the challenges and strains it puts on herself and other Android Women:
``Run on for your life or you can dance you can die // She's praying in the sand like she's the last samurai" \cite{danceordie}.
Mayweather struggles with the realities of being marginalized while expected to lead a \emph{Cold War}:
``I'm trying to find my peace.  I was made to believe theres something wrong with me!" \cite{coldwar}

Finally, an important discussion about abusive relationships is started in the second suite, but left deliberately open, for resolution in the next suite:
``I'm locked inside a land called foolish pride // where the man is always right // he hates to talk but loves to fight // is that alright?" \cite{lockedinside}

%------------------------------------------------

\section*{The ArchAndroid Suite III: Naming Love and its Limits}

Cindi Mayweather is awakened to her world by the consequences of discovering and creating love.
In the third suite of \emph{Metropolis}, Mayweather grapples with her love for Anthony Greendown in moving ballads, careful considerations and unspoken understandings.
Though Mayweather has sacrificed for their relationship, and knows of its power and beauty, though she is madly in love with Greendown, most of the third suite is about grappling with the limits of that love. 

Mon\'ae begins the suite with a discussion about sex, both in terms of sexual awakening and sexual deviancy.
In \emph{Wondaland}, Mon\'ae's singsong narrator implores: ``Take her back to wondaland, she thinks she left her underpants" \cite{wondaland}.
In \emph{Make The Bus}, Mayweather is confessing that she has ideas and temptations that she feels she ought avoid:
``I've got this terrible fixation // ... // just want to keep it in the realm of fantasy" \cite{makethebus}.
By starting with open and honest, unapologetic discussions on sex (``This is my life. We belong here, stay the night." \cite{wondaland}) Mon\'ae centers sex as both a natural part of a young black woman's journey, and sets the stage to discuss its abuse and mistreatment.

In \emph{Neon Valley Street} (my unequivocal favorite of any of Mon\'ae's works, by the way), Mayweather reminisces on her and Greendown's past (on the street where she had lived before her fugitive life), and the elaborate, intricate, careful detail she has placed in to her affection and care reverberates through her voice into hearts, minds and guts.
``May this song reach your heart // may your ears love the sweet melody // Every note, every chord // I've arranged it for you and for me'' \cite{neonvalleystreet}.
Love's rarity, beauty and intimacy is on display in the third suite, but attached to it are questions of what love can look like, and how one performs it.
In \emph{Say You'll Go}, Mayweather implores her partner to be willing to pursue love with the knowledge of its challenges and scarcity:
``Love is such a novelty // A rarely painted masterpiece // A place few people go or ever know" \cite{sayyouwillgo}.

But Mayweather knows that in asking her partner to `say he'll go', there is a power question associated, one of ``Who will lead? Who will follow?" \cite{sayyouwillgo}, echoing bell hooks' language on power, love, and trust discussed above \cite{newvisions}.
Power and love are frequently at play in \emph{Metropolis}, and from what little we have of their relationship, it seems as if Mayweather is calling the shots, while Greendown has the agency to execute on them because of his relative power as a human: ``Sir Greendown // Let's leave in an hour // Meet me at the tower // Ride your horse" \cite{greendown}.
This power imbalance is always at the heart of her questions on love, and hooks, as mentioned before, questions whether love can ever truly exist as a process between people with any sort of power over one another \cite{newvisions}.

\emph{All About Love: New Visions} also presents us with a question about how Mayweather (and Mon\'ae) perform and describe love.
Throughout the five suites of \emph{Metropolis}, Mon\'ae uses the word love as hooks would, as a verb (though ignoring hooks' call for its sparing use).
Hooks impresses upon her readers the importance of viewing love as an active process and series of states of being, not a commodity `like dixie cups' which can be attained and lost:
``How different things might be if, rather than saying `I think I'm in love,' we were saying `I've connected with someone in a way that makes me think I'm on the way to knowing love.' Or if instead of saying `I am in love' we say `I am loving' or `I will love'. Our patterns around romantic love are unlikely to change if we do not change our language.'' \cite{newvisions}.
Similarly, Mon\'ae is careful to avoid using love as an object, treating it alternatingly as a verb and as an objective: ``Tell me are you bold enough to reach for love?" \cite{manymoons}.

The penultimate track of the third suite is the rich and long 57821, named after Cindi Mayweather's system assigned serial ID number.
In the Deep Cotton featured song, there is storytelling craft buried beneath a choral masterpiece, describing Cindi's imprisonment, her longing for her beloved, and a carefully wrought connection between the love that she has developed, built and felt longing for, and an idea of her deification.
In the song, there is the introduction of explicit language around `the One', or `the ArchAndroid', a figure we later learn is a feature of android mythology, a revolutionary who will bring humans and androids together as one.
This idea of reunification is intertwined with Mayweathers captivity and longing for her beloved: is she the one who can unify humans and androids? Is she the one who can create and maintain love with Anthony?

The second of these questions is answered in the final song of the suite, BabopbyeYa, where we get the allusion to domestic abuse as a feature of the relationship between Mayweather and Greendown.
Mayweather spends the first three quarters of the song (which reads like a broadway power ballad) describing her layers of infatuation, admiration and pleasure that she gets from Greendown.
However, the song crecendos to a cool peak, where Mayweather offers us the only hint to the challenges her relathionship has faced: ``This time I shall be unafraid // And violence will not move me" \cite{babopbyeya}.
Having been trained in recognizing patterns of partner abuse, this line deeply concerns me.
On again, off again relationships with mutual co-dependance (`this time'), and a history of violent interactions ('violence will not move me') are both red flags for partner abuse predicated on power imbalances.
With this one line, Mon\'ae paints us a different picture, one that alludes to the existence of domestic violence as a problem within black communities, and simultaneously acting as example in how difficult it can be to name and discuss domestic violence openly. 

The question of whether or not to discuss domestic violence within the Black community had been a debate of the second wave of black feminism.
Many were perturbed to see Toni Morrison portraying rape and incest as emblematic horrors in \emph{The Bluest Eye}, which many criticized for its role in degrading the `strength' of the black community.
This shows the flaws in the the thinking of the time, and the way in which Black Women's narratives and opportunities were put as tertiary objectives behind those of black men.
The tangible break between the second and third wave of black feminism lies in this space, in this question of prioritization, self advocacy, and sexual violence.

In \emph{Outlaw Culture: Resisting Representations}, bell hooks gives us the language to describe how domestic abuse as a mechanism of power is perpetuated by the `seduction of violence', from which the tenth chapter gains its name \cite{hooks1994outlaw}.
Hooks shows us how because Black men are systematically denied access to power in a white society, that sexual violence is the manifestation of power that is most available to them, and this garners and support a culture of rape and partner abuse.
Hooks discusses how the `seduction' of any power gained through the exploitation of others needs to be rejected to move forward, and that women can change this pattern by refusing to be seduced by power, and specifically, violence.

In Cindi's words, she has made a deliberate choice: she will no longer be moved (or seduced) through violence. 
Without this feature of her life, she sees ``beyond tomorrow // This life of strife and sorrow // My freedom calls and I must go." \cite{babopbyeya}
Cindi Mayweather chooses her freedom from an abusive relationship over the pleasure that the relationship has brought her.
That choice launches her into the fourth suite: though she is no longer fighting for her own love, she has been awakened to fight for so much more, she has fully transitioned into the third wave of Black Feminism.

%------------------------------------------------

\section*{The Electric Lady Suite IV: Queerness, Performance and Activism}

It is tough to write about the fourth suite of \emph{Metropolis} without singularly centering the Black Feminist Anthem Q.U.E.E.N, but we will try.
In the fourth suite, Monae articulates a clear set of guidelines for activism, implores her audience to act against oppression, and begins honest conversations about queerness.


``They keep us underground working hard for the greedy, but when it's time pay they turn around and call us needy." \cite{queen}

- Rejection of Violence, unite in love
``I use my words when stones come around.'' \cite{manymoons}
In an interlude, `Bop Bot Say What' proposes violence as a means to redress the agreed upon inconsistiences of ruling logic: ``hitting us all up in the head and wondering why we don't think straight" \cite{chromeshoppe}.
``I'm praying for the man sitting without much time // May he understand the clock will never rewind // Wisen him and sharpen him and give him a motto // hate no more, said he must hate no more" \cite{danceordie}
``Sudanese and Congolese who put the roll in the rock // From here to Sudan, Metropolis to Iraq" \cite{danceordie}

- Activism
``March to the streets 'cause I'm willing and I'm able" \cite{queen}
``Wake up Mary" \cite{sallyride}
``Melanie 45221, Assatta, 8550." \cite{chromeshoppe}
``Keep leading like a young Harriet Tubman" \cite{queen}
``Day dreamers please wake up, we can't sleep no more." \cite{sincerelyjane}
``Electric Ladies will you sleep? Or will you preach?" \cite{queen}

Media as a form of Power:
...no matter how sophisticated our strategies of critique and intervention, [we] are usually seduced, at least for a time, by the images we see on the screen. They have power over us, and we have no power over them - hooks, media theory

- Queerness
``Is it peculiar to like the way she wears her tights?" \cite{queen}
``Would your god accept me in my black and white? Would he approve the way I'm made? Or should I reprogram reprogram and get down?!?" \cite{queen}

%------------------------------------------------

\section*{The Electric Lady Suite V: Dissemblance, Self Empowerment and Victory}

Empowerment: 
``Oh make it rain, ain't a thang and the sky to fall (The silver bullet's in your hand and the war's heating up)" \cite{manymoons}
``She'd keep it to herself and nobody could understand her // Even when she thought that she couldn't she carried on // She couldn't imagine both of her daughters here all alone" \cite{ghettowoman}
``I wish they could just realize // That all you've ever needed was someone to free your mind // Carry on Ghetto Woman // Cause even in your darkest hours I still see your light" \cite{ghettowoman}

Victory, Struggle, Empowerment and Energy
``To be victorious, you must find glory in the little things" \cite{victory}
``Cause when the rain falls // My seed will grow // I'll be further to my dreams tomorrow" \cite{victory}


Non-reliance on Dieties:
``Indivisible Sum // Here's the book, now the saga's begun" \cite{fiveseveneighttwoone}
``A long long way to find the one // We'll keep on dancing till she comes // These dreams are forever" \cite{danceordie}
``I want to know if you believe Cindi Mayweather is not only just electric lady, number one and all, but also the ArchAndroid?" \cite{favoritefugitive}
``Heaven is betting on us." \cite{primetime}
``Take me to the river, my soul is looking for a word from God. Oh God, like a rose in the cold, will I rise?" \cite{sallyride}
Ending on a celebration of her unique features \cite{dandridgeeyes}, and a party \cite{whatanexperience}

%------------------------------------------------

\section*{Meta Analysis: Self Naming, Performance, Epistomology}

- Self Naming and categorization
``She who writes the movie owns the script and the sequel." \cite{queen}
Labels, Categories
Mon\'ae has frequently worked with an Atlanta based group, Jaspects.
Their collaborations, including \emph{My First Love}, \emph{Peachtree Blues} and \emph{2012}, feature Mon\'ae's stunning voice over the band's signature modern jazz sound \cite{peachtreeblues} \cite{myfirstlove}.
In her 2009 collaboration with the Atlanta group Jaspects, Mon\'ae's song 2012 implored listeners to ask what freedom means to them, and how they actualize it for themselves and for others.
Though she is not featured on the title song of the album ``The Polkadotted Stripe", freedom within the context of the album revolves around the idea of false definitions.
``We see black or white, polkadots or stripes. Freedom we deny, lets open our eyes." \cite{polkadottedstripe}
Mon\'ae questions dichotomies and categorization throughout the Metropolis suite, through explict and subtle means.
Though not always as clear as in Q.U.E.E.N. (``Categorize me, I defy Every Label" \cite{queen}), 
On her own style, Mon\'ae refuses to adhere to categorization which undermines the potency of her work:
``We?re growing up in the iPod generation in terms of genres. I do away with labels and categories. I don't believe in them ... That was created by man." \cite{joeyguerra2010}


``Categorize me, I defy every label." \cite{queen}
``I Just want to be, as free as I can be." \cite{polkadottedstripe}
``We see black or white, polkadots or stripes. Freedom we deny, lets open our eyes." \cite{polkadottedstripe}
``Additive models of oppression are firmly rooted in the either/or dichotomous thinking of Eurocentric, masculinist thought. One must be either Black or white in such thought systems--persons of ambiguous racial and ethnic identity constantly battle with questions such as `what are your, anyway?' This emphasis on quantification and categorization occurs in conjunction with the belief that either/or categories must be ranked. The search for certainty of this sort requires that one side of a dichotomy be privileged while its other is denigrated. Privilege becomes defined in relation to its other." \cite{collins2009}


- Interactions with Whiteness
ALL OF \cite{favoritefugitive}.

- Episotomology
``Add us to equations but they'll never make us equal." \cite{queen}
``Because the booty don't lie" \cite{queen}

- Calls to Wake up
``Day dreamers please wake up, we can't sleep no more." \cite{sincerelyjane}
``Electric Ladies will you sleep? Or will you preach?" \cite{queen}
``Just wake up, Mary, Have you heard the news? Oh, just wake up, Mary, You got the right to choose" \cite{sallyride}


%------------------------------------------------


\section*{Misc/Unassigned}

BOOK IDEAS:

HOOKS - Outlaw culture, resisting representations

"The true focus of revolutionary change is never merely the oppressive situations which we seek to escape, but that piece of the oppressor which is planted deep within each of us." \cite{lordredefiningdifference} 

``What lis between the lines are the things that women of color do not tell each other. There are reasons for our silences: the change in generation between mother and daughter, the language barriers between us, our sexual identity, the educational opportunities we had or missed, the specific cultural history of our race, the physical conditions of our bodies and the labor... We begin by speaking directly to the deaths and disappointments. Here we begin to fill in the spaces of silence between us. For it is between these seemingly irreconcilable lines--the class lines, the politically correct lines, the daily lines we run down to each other to keep difference and desire at a distance--that the truth of our connection lies." \cite{thebridge}

``As women we have been taught to either ignore our differences or to view them as causes for separation and suspicion rather than as forces for change. Without community, there is no liberation, only the most vulnerable and temporary armistice between an individual and her oppression. But community must not mean a shedding of our differences, not the pathetic pretense that these differences do not exist." \cite{lorde2003master}

``Kids are killing kids and then the kids join the army" \cite{danceordie}

In their 2002 work on how girls of color can be intentionally educated and counseled to thrive in systems which frequently invalidate their experiences and voice, Iglacias and Cormer discuss the adolescent transition as one that is rooted in struggle for girls of color, and that ``A girl who will develop as a woman who embraces freedom has a strong sense of self-confidence and self-efficacy.'' \cite{iglesiascormier}

?If any female feels she need anything beyond herself to legitimate and validate her existence, she is already giving away her power to be self-defining, her agency.? \cite{hooksforeveryone}

?All too often women believe it is a sign of commitment, an expression of love, to endure unkindness or cruelty, to forgive and forget. In actuality, when we love rightly we know that the healthy, loving response to cruelty and abuse is putting ourselves out of harm's way.? \cite{newvisions}


%----------------------------------------------------------------------------------------
%	BIBLIOGRAPHY
%----------------------------------------------------------------------------------------

%\nocite{*}
\bibliography{ArchAndroid,Metropolis,ElectricLady,ClassReading,External}{}
\bibliographystyle{unsrt}

%----------------------------------------------------------------------------------------

\end{document}